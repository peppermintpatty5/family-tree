\documentclass[letterpaper]{article}
\usepackage{amsmath}
\usepackage{amssymb}
\usepackage{array}
\usepackage{booktabs}
\usepackage[T1]{fontenc}
\usepackage[indent]{parskip}


\title{Family Trees}
\author{Andrew Kritzler}

\begin{document}
\maketitle

\section{Data Model}

Define \(P = M \cup F\) to be the set of all people where \(M\) and \(F\) are the sets of all males and females respectively.
\(M \cap F = \varnothing{}\) since no person can be both male and female.
The functions \(m : P \to F\) and \(f : P \to M\) map people to their mothers and fathers respectively.
Repeated compositions of \(m\) and \(f\) denote various ancestors, e.g., \(m \circ f\) reads as ``mother of father''.

Take the following example of some characters from \textit{The Simpsons}:
\begin{align*}
    M & = \{\text{Abe}, \text{Bart}, \text{Clancy}, \text{Herb}, \text{Homer}\},                                                 \\
    F & = \{\text{Gaby}, \text{Jacqueline}, \text{Lisa}, \text{Maggie}, \text{Marge}, \text{Mona}, \text{Patty}, \text{Selma}\}.
\end{align*}
Immediately there is a problem of undefined values for \(m\) and \(f\).
Who are the parents of Abe ``Grampa'' Simpson?
And who are their parents, \dots?
The assumption that every person has a mother and father creates the same dilemma at the chicken and the egg.
If we believe this assumption to be true, then one (or both) of the following impossible conclusions would result:
\begin{enumerate}
    \item Infinitely many males and females have existed over the course of human history:
          \[
              |M| = |F| = |\mathbb{N}|.
          \]
    \item Human ancestry contains loops, i.e., someone is their own ancestor. Formally, following some sequence of mothers and fathers from some person \(p\) will lead back to \(p\):
          \[
              (\exists p \in P)
              (
              (\exists (\phi_1, \phi_2, \dotsc, \phi_n) \in {\{m, f\}}^n)
              ( (\phi_1 \circ \phi_2 \circ \cdots \circ \phi_n)(p) = p)
              ).
          \]
\end{enumerate}

Philosophical discussion aside, let us remedy this problem by allowing loops and adding the special values \(\mathcal{A} \in M\) and \(\mathcal{E} \in F\) (to represent ``Adam and Eve'').
Additionally, define their parents to be themselves, i.e., \(m(\mathcal{A}) = m(\mathcal{E}) = \mathcal{E}\) and \(f(\mathcal{A}) = f(\mathcal{E}) = \mathcal{A}\).
Now the ancestry can be constructed as shown in Table~\ref{table:simpsons}.

\begin{table}
    \centering
    \begin{tabular}{*{3}{m{5em}}}
        \toprule
        \(p \in P\)     & \(m(p)\)        & \(f(p)\)        \\
        \midrule
        \(\mathcal{A}\) & \(\mathcal{E}\) & \(\mathcal{A}\) \\
        \(\mathcal{E}\) & \(\mathcal{E}\) & \(\mathcal{A}\) \\
        Abe             & \(\mathcal{E}\) & \(\mathcal{A}\) \\
        Bart            & Marge           & Homer           \\
        Clancy          & \(\mathcal{E}\) & \(\mathcal{A}\) \\
        Gaby            & \(\mathcal{E}\) & \(\mathcal{A}\) \\
        Herb            & Gaby            & Abe             \\
        Homer           & Mona            & Abe             \\
        Jacqueline      & \(\mathcal{E}\) & \(\mathcal{A}\) \\
        Lisa            & Marge           & Homer           \\
        Maggie          & Marge           & Homer           \\
        Marge           & Jacqueline      & Clancy          \\
        Mona            & \(\mathcal{E}\) & \(\mathcal{A}\) \\
        Patty           & Jacqueline      & Clancy          \\
        Selma           & Jacqueline      & Clancy          \\
        \bottomrule
    \end{tabular}
    \caption{Ancestry example of \textit{The Simpsons}}\label{table:simpsons}
\end{table}

\section{Relationships}

Family relationships are defined by the nearest common ancestors.
For example, siblings \(x\) and \(y\) have the same mother and father:
\[
    m(x) = m(y) \land f(x) = f(y).
\]
By this definition, however, Abe and Mona would mistakenly be considered siblings as shown in Table~\ref{table:simpsons}.
Since \(\mathcal{A}\) and \(\mathcal{E}\) are meant as null values, let us define the equivalence relations \({\sim_m}\), \({\sim_f}\), and \({\sim}\) on \(P\) that check for two people having the same, \textit{not-null} mother and father:
\begin{align*}
    {\sim_m} & = \{(x, y) \mid x = y \lor (m(x) = m(y) \land m(x) \neq \mathcal{E})\}, \\
    {\sim_f} & = \{(x, y) \mid x = y \lor (f(x) = f(y) \land f(x) \neq \mathcal{A})\}, \\
    {\sim}   & = \{(x, y) \mid x \sim_m y \land x \sim_f y\}.
\end{align*}
Now the previous definition for siblings can be corrected to \(x \sim y\), giving the expected result of \(\text{Bart} \sim \text{Lisa} \sim \text{Maggie}\) and \(\text{Abe} \nsim \text{Mona}\).

Reflexivity, along with symmetry and transitivity, is one of the three properties of equivalence relations.
It requires that \((\forall x \in P)(x \sim x)\).
This idea of every person being their own sibling may seem strange but should make sense when considering equivalence classes.
For example:
\begin{align*}
    [\text{Bart}] & = \{x \in P \mid \text{Bart} \sim x\}          \\
                  & = \{\text{Bart}, \text{Lisa}, \text{Maggie}\}.
\end{align*}

\end{document}
