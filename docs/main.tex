\documentclass{article}
\usepackage{amsmath}
\usepackage{amssymb}
\usepackage{booktabs}
\usepackage[T1]{fontenc}
\usepackage[indent]{parskip}
\usepackage{wasysym}


\title{Family Trees}
\author{Andrew Kritzler}

\begin{document}
\maketitle

\section{Data Model}

Suppose \(P = M \cup F\) is the set of all people where \(M\) and \(F\) are the sets of all males and females respectively.
\(M \cap F = \varnothing{}\) since no person can be both male and female.
The functions \(m : P \to F\) and \(f : P \to M\) map people to their mothers and fathers respectively.

Take the following example of some characters from \textit{The Simpsons}, where Homer and Marge are the parents of Bart, Lisa, and Maggie:
\[
    \begin{aligned}
        M & = \{\text{Homer}, \text{Bart}\},                \\
        F & = \{\text{Marge}, \text{Lisa}, \text{Maggie}\}.
    \end{aligned}
\]
Immediately there is a critical problem: who are the parents of Homer and Marge?
In fact, this problem is unending as there will always be some person without parents regardless of how many people are added to \(P\).
By requiring every person to have a mother and father, at least one of these seemingly impossible scenarios must be true:
\begin{enumerate}
    \item Infinitely many males and females have existed over the course of human history:
          \[
              |M| = |F| = |\mathbb{N}|.
          \]
    \item Human ancestry contains loops, i.e., someone is their own ancestor. Formally, following some sequence of mothers and fathers from some person \(p\) will lead back to \(p\):
          \[
              (\exists p \in P)
              (
              (\exists (\phi_1, \phi_2, \dotsc, \phi_n) \in {\{m, f\}}^n)
              ( (\phi_1 \circ \phi_2 \circ \cdots \circ \phi_n)(p) = p)
              ).
          \]
\end{enumerate}

Philosophical discussion aside, let us remedy this problem by allowing loops and adding the special values \(\mathcal{A} \in M\) and \(\mathcal{E} \in F\) (to represent ``Adam and Eve'').
Additionally, define their parents to be themselves, i.e., \(m(\mathcal{A}) = m(\mathcal{E}) = \mathcal{E}\) and \(f(\mathcal{A}) = f(\mathcal{E}) = \mathcal{A}\).
Now the ancestry can be constructed as shown in Table~\ref{table:simpsons}.

\begin{table}
    \centering
    \begin{tabular}{lll}
        \toprule
        \(p \in P\)     & \(m(p)\)        & \(f(p)\)        \\
        \midrule
        \(\mathcal{A}\) & \(\mathcal{E}\) & \(\mathcal{A}\) \\
        \(\mathcal{E}\) & \(\mathcal{E}\) & \(\mathcal{A}\) \\
        Homer           & \(\mathcal{E}\) & \(\mathcal{A}\) \\
        Marge           & \(\mathcal{E}\) & \(\mathcal{A}\) \\
        Bart            & Marge           & Homer           \\
        Lisa            & Marge           & Homer           \\
        Maggie          & Marge           & Homer           \\
        \bottomrule
    \end{tabular}
    \caption{Ancestry example of \textit{The Simpsons}}\label{table:simpsons}
\end{table}

\section{Relationships}

Family relationships are defined by the nearest common ancestors.
For example, siblings \(x\) and \(y\) have the same mother and father:
\[
    m(x) = m(y) \land f(x) = f(y).
\]
By this definition, however, Homer and Marge would be considered siblings as shown in Table~\ref{table:simpsons}.
Since \(\mathcal{A}\) and \(\mathcal{E}\) are intended to be used as null values, let us define the following equivalence relations to specifically exclude cases like this:
\[
    \begin{aligned}
        {\overset{m}{\sim}} & = \{(x, y) : m(x) = m(y) \land m(x) \neq \mathcal{E}\}, \\
        {\overset{f}{\sim}} & = \{(x, y) : f(x) = f(y) \land f(x) \neq \mathcal{A}\}.
    \end{aligned}
\]
Now the previous definition for siblings shall be redefined as:
\[
    x \overset{m}{\sim} y \land x \overset{f}{\sim} y.
\]

\end{document}
