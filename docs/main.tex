\documentclass[letterpaper,11pt]{article}
\usepackage{amsmath}
\usepackage{amssymb}
\usepackage{array}
\usepackage{booktabs}
\usepackage[labelfont=bf]{caption}
\usepackage[T1]{fontenc}
\usepackage[indent]{parskip}
\usepackage{wasysym}


\title{Family Trees}
\author{Andrew Kritzler}

\begin{document}
\maketitle

\section{Data Model}

Define the set of all people \(P = \male \cup \female\) to be the union of all males and females such that \(\male \cap \female = \varnothing\).
The functions \(f : P \to \male\) and \(m : P \to \female\) shall map people to their fathers and mothers respectively.
Table~\ref{table:simpsons} shows an example using the following characters from \textit{The Simpsons}:
\begin{align*}
    \male   & = \{\text{Abe}, \text{Bart}, \text{Clancy}, \text{Herb}, \text{Homer}\},                                \\
    \female & = \{\text{Jackie}, \text{Lisa}, \text{Maggie}, \text{Marge}, \text{Mona}, \text{Patty}, \text{Selma}\}.
\end{align*}

In practice there will always be someone with an unknown mother or father.
However, our model requires \(f\) and \(m\) to be defined for all values of \(P\).
To correct this problem let's introduce special ``Adam and Eve'' values \(\mathcal{A} \in \male\) and \(\mathcal{E} \in \female\) where:
\begin{align*}
    \mathcal{A} & = f(\mathcal{A}) = f(\mathcal{E}), \\
    \mathcal{E} & = m(\mathcal{A}) = m(\mathcal{E}).
\end{align*}

\begin{table}
    \centering
    \begin{tabular}{*{4}{l}}
        \toprule
        \(p \in P\) & \(f(p)\)        & \(m(p)\)        & \(\alpha(p)\)                                                                             \\
        \midrule
        Abe         & \(\mathcal{A}\) & \(\mathcal{E}\) & \(\varnothing\)                                                                           \\
        Bart        & Homer           & Marge           & \(\{\text{Homer}, \text{Marge}, \text{Abe}, \text{Mona}, \text{Clancy}, \text{Jackie}\}\) \\
        Clancy      & \(\mathcal{A}\) & \(\mathcal{E}\) & \(\varnothing\)                                                                           \\
        Herb        & Abe             & \(\mathcal{E}\) & \(\{\text{Abe}\}\)                                                                        \\
        Homer       & Abe             & Mona            & \(\{\text{Abe}, \text{Mona}\}\)                                                           \\
        Jackie      & \(\mathcal{A}\) & \(\mathcal{E}\) & \(\varnothing\)                                                                           \\
        Lisa        & Homer           & Marge           & \(\{\text{Homer}, \text{Marge}, \text{Abe}, \text{Mona}, \text{Clancy}, \text{Jackie}\}\) \\
        Maggie      & Homer           & Marge           & \(\{\text{Homer}, \text{Marge}, \text{Abe}, \text{Mona}, \text{Clancy}, \text{Jackie}\}\) \\
        Marge       & Clancy          & Jackie          & \(\{\text{Clancy}, \text{Jackie}\}\)                                                      \\
        Mona        & \(\mathcal{A}\) & \(\mathcal{E}\) & \(\varnothing\)                                                                           \\
        Patty       & Clancy          & Jackie          & \(\{\text{Clancy}, \text{Jackie}\}\)                                                      \\
        Selma       & Clancy          & Jackie          & \(\{\text{Clancy}, \text{Jackie}\}\)                                                      \\
        \bottomrule
    \end{tabular}
    \caption{Ancestry example of \textit{The Simpsons}}\label{table:simpsons}
\end{table}

\section{Genetic relationships}

Let \(\alpha : P \to \mathcal{P}(P)\):
\[
    \alpha(p) = \begin{cases}
        \varnothing                                                                                 & p \in \{\mathcal{A}, \mathcal{E}\}    \\
        (\{f(p), m(p)\} \setminus \{\mathcal{A}, \mathcal{E}\}) \cup \alpha(m(p)) \cup \alpha(f(p)) & p \notin \{\mathcal{A}, \mathcal{E}\}
    \end{cases}
\]

Genetic relationships are defined by common ancestors.
Bart, Lisa, and Maggie are related to Aunt Patty and Selma by Marge's parents.
This (2, 1) pattern of going two generations up and one generation down uniquely defines the uncle/aunt relationship.
Uncle Herb is similarly related by Homer's father, making him their half-uncle.

siblings \(x\) and \(y\) have the same mother and father:
\[
    m(x) = m(y) \land f(x) = f(y).
\]
By this definition, however, Abe and Mona would mistakenly be considered siblings as shown in Table~\ref{table:simpsons}.
Since \(\mathcal{A}\) and \(\mathcal{E}\) are meant as null values, let us define the equivalence relations \({\sim_m}\), \({\sim_f}\), and \({\sim}\) on \(P\) that check for two people having the same, \textit{not-null} mother and father:
\begin{align*}
    {\sim_m} & = \{(x, y) \mid x = y \lor (m(x) = m(y) \land m(x) \neq \mathcal{E})\}, \\
    {\sim_f} & = \{(x, y) \mid x = y \lor (f(x) = f(y) \land f(x) \neq \mathcal{A})\}, \\
    {\sim}   & = \{(x, y) \mid x \sim_m y \land x \sim_f y\}.
\end{align*}
Now the previous definition for siblings can be corrected to \(x \sim y\), giving the expected result of \(\text{Bart} \sim \text{Lisa} \sim \text{Maggie}\) and \(\text{Abe} \nsim \text{Mona}\).


\end{document}
